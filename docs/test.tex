\documentclass[spanish,a4paper]{book}
\usepackage{lmodern}
\usepackage{amssymb,amsmath}
\usepackage{ifxetex,ifluatex}
\usepackage{fixltx2e} % provides \textsubscript
\ifnum 0\ifxetex 1\fi\ifluatex 1\fi=0 % if pdftex
  \usepackage[T1]{fontenc}
  \usepackage[utf8]{inputenc}
\else % if luatex or xelatex
  \ifxetex
    \usepackage{mathspec}
  \else
    \usepackage{fontspec}
  \fi
  \defaultfontfeatures{Ligatures=TeX,Scale=MatchLowercase}
\fi
% use upquote if available, for straight quotes in verbatim environments
\IfFileExists{upquote.sty}{\usepackage{upquote}}{}
% use microtype if available
\IfFileExists{microtype.sty}{%
\usepackage{microtype}
\UseMicrotypeSet[protrusion]{basicmath} % disable protrusion for tt fonts
}{}
\usepackage[inner = 5cm, outer = 5cm, top = 2.5cm, bottom = 2.5cm]{geometry}
\usepackage{hyperref}
\hypersetup{unicode=true,
            pdftitle={Tesis},
            pdfauthor={Elio Campitelli},
            pdfborder={0 0 0},
            breaklinks=true}
\urlstyle{same}  % don't use monospace font for urls
\ifnum 0\ifxetex 1\fi\ifluatex 1\fi=0 % if pdftex
  \usepackage[shorthands=off,main=spanish]{babel}
\else
  \usepackage{polyglossia}
  \setmainlanguage[]{spanish}
\fi
\usepackage{graphicx,grffile}
\makeatletter
\def\maxwidth{\ifdim\Gin@nat@width>\linewidth\linewidth\else\Gin@nat@width\fi}
\def\maxheight{\ifdim\Gin@nat@height>\textheight\textheight\else\Gin@nat@height\fi}
\makeatother
% Scale images if necessary, so that they will not overflow the page
% margins by default, and it is still possible to overwrite the defaults
% using explicit options in \includegraphics[width, height, ...]{}
\setkeys{Gin}{width=\maxwidth,height=\maxheight,keepaspectratio}
\IfFileExists{parskip.sty}{%
\usepackage{parskip}
}{% else
\setlength{\parindent}{0pt}
\setlength{\parskip}{6pt plus 2pt minus 1pt}
}
\setlength{\emergencystretch}{3em}  % prevent overfull lines
\providecommand{\tightlist}{%
  \setlength{\itemsep}{0pt}\setlength{\parskip}{0pt}}
\setcounter{secnumdepth}{5}
% Redefines (sub)paragraphs to behave more like sections
\ifx\paragraph\undefined\else
\let\oldparagraph\paragraph
\renewcommand{\paragraph}[1]{\oldparagraph{#1}\mbox{}}
\fi
\ifx\subparagraph\undefined\else
\let\oldsubparagraph\subparagraph
\renewcommand{\subparagraph}[1]{\oldsubparagraph{#1}\mbox{}}
\fi

%%% Use protect on footnotes to avoid problems with footnotes in titles
\let\rmarkdownfootnote\footnote%
\def\footnote{\protect\rmarkdownfootnote}

%%% Change title format to be more compact
\usepackage{titling}

% Create subtitle command for use in maketitle
\newcommand{\subtitle}[1]{
  \posttitle{
    \begin{center}\large#1\end{center}
    }
}

\setlength{\droptitle}{-2em}
  \title{Tesis}
  \pretitle{\vspace{\droptitle}\centering\huge}
  \posttitle{\par}
\subtitle{Una tesis}
  \author{Elio Campitelli}
  \preauthor{\centering\large\emph}
  \postauthor{\par}
  \date{}
  \predate{}\postdate{}

\linespread{1.25}
\usepackage{subfig}
\usepackage{hyperref}
\usepackage{marginnote}
\usepackage[nomarkers,figuresonly]{endfloat}
\usepackage{pdflscape}
\DeclareDelayedFloatFlavor{landscape}{figure}
\usepackage[spanish]{todonotes}
\usepackage{wrapfig}

\begin{document}
\maketitle

{
\setcounter{tocdepth}{3}
\tableofcontents
}
Resumen.

\todo[inline]{Listado de abreviaturas}
\todo[inline]{Revisar TODOS los epígrafes}

\chapter{Introducción}\label{introduccion}

\begin{itemize}
\tightlist
\item
  Antecedentes\\
  Además de lo que hay en lo de las becas + lo que fui encontrando,
  agregar sobre las climatologías disponibles y sus limitaciones.
\item
  Objetivo General
\item
  Objetivo particular
\end{itemize}

Esto es para probar una referencia bibliográfica: Vera et~al.
(\protect\hyperlink{ref-Vera2004}{2004}) y (Vera et~al.
\protect\hyperlink{ref-Vera2004}{2004})

\chapter{Métodos y Materiales}\label{metodos-y-materiales}

\todo[inline]{Agregar en algún lugar algo sobre las estadísticas circulares}

\section{Conceptos básicos}\label{conceptos-basicos}

\begin{itemize}
\tightlist
\item
  Ondas cuasiestacionarias
\item
  fourier
\item
  wavelets
\item
  Flujo de actividad de onda.
\end{itemize}

\todo[inline]{chequear este paper: https://link.springer.com/article/10.1007/s00024-012-0635-9}

Ejemplo:

\begin{figure*}
\includegraphics[width=150mm, ]{fig/tesis/fourier-ejemplo-1} \caption{Ejemplo fourier - fig:fourier-ejemplo}\label{fig:fourier-ejemplo}
\end{figure*}

Cosas para ver de \autoref{fig:fourier-ejemplo}:\\
la @ref(fig:fourier-ejemplo) Descripción del ``rol'' de cada número de
onda en generar el campo final. La QS1 es la principal, marcando altas
presiones al sur del pacífico y bajas al sur de África. La onda 3
modifica ese patrón simple haciendo que los máximos y mínimos no sean
continuos.

\begin{itemize}
\tightlist
\item
  Wavelets
\end{itemize}

\begin{figure*}
\subfloat[Campo\label{fig:wavelet-ejemplo1}]{\includegraphics[width=75mm, ]{fig/tesis/wavelet-ejemplo-1} }\subfloat[Análisis de Wavelets en círculos de latitud señalados.\label{fig:wavelet-ejemplo2}]{\includegraphics[width=75mm, ]{fig/tesis/wavelet-ejemplo-2} }\caption{Wavelets - fig:wavelet-ejemplo}\label{fig:wavelet-ejemplo}
\end{figure*}

Cosas para ver:\\
Cambio en el máximo. Localización en vez de un número para cada latitud.

\section{Fuentes de datos}\label{fuentes-de-datos}

\section{Descripción de SPEEDY}\label{descripcion-de-speedy}

\chapter{Climatología observada}\label{climatologia-observada}

En esta sección se presentan campos medios y anomalías zonales de altura
geopotencial, temperatura, viento zonal, viento meridinal, gradiente
meridional de vorticidad absoluta y el número de onda estacionario, y
función corriente como introducción general al estado medio de la
atmósfera sobre el cual se desarrollan las ondas estacionarias. Luego se
analizan los campos de amplitud y varianza explicada por las ondas
cuasiestacionarias (QS) en sí mismas.

\section{Altura geopotencial}\label{altura-geopotencial}

Campo medio:

\begin{landscape}\begin{figure}

{\centering \includegraphics[width=227mm,]{fig/tesis/gh-ncep-1} 

}

\caption{Campo de Z (NCEP) - fig:gh-ncep}\label{fig:gh-ncep}
\end{figure}
\end{landscape}

\hypertarget{refs}{}
\hypertarget{ref-Vera2004}{}
Vera, Carolina, Gabriel Silvestri, Vicente Barros, y Andrea Carril.
2004. «Differences in El Niño response over the Southern Hemisphere».
\emph{Journal of Climate} 17 (9): 1741-53.
doi:\href{https://doi.org/10.1175/1520-0442(2004)017\%3C1741:DIENRO\%3E2.0.CO;2}{10.1175/1520-0442(2004)017\textless{}1741:DIENRO\textgreater{}2.0.CO;2}.


\end{document}
